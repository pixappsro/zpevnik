\begin{TEXT}{Strom}
\SLOKA \Ch{a}{Polní} cestou kráčeli šuma\Ch{G}{ři} do vísky hrát \NL
\Ch{a}{svatby} pohřby tahle cesta po\Ch{G}{znala} tolikrát \NL
Po \Ch{F}{jedné} svatbě se \Ch{G}{chudým} lidem \Ch{C}{synek} naro\Ch{a}{dil} \NL
a \Ch{F}{táta} mu u \Ch{G}{prašný} cesty \Ch{E}{života} strom zasadil 
\REFREN  A on tam \Ch{A}{stál} a koukal \Ch{f#}{do polí} \NL
byl jak \Ch{D}{král} sám v celém \Ch{E}{okolí} \NL
korunu \Ch{A}{měl}, korunu měl, i když ne \Ch{D}{ze} zlata  \NL
a jeho \Ch{A}{pokladem} byla \Ch{E}{tráva} střapa\Ch{A}{tá} 
\SLOKA Léta běží a na ten příběh už si nikdo nevzpomněl \NL
jen košatý strom se u cesty ve větru tiše chvěl \NL
a z vísky bylo město a to město začlo chtít \NL
asfaltovej koberec až na náměstí mít 
\SLOKA Že strom byl v cestě plánované to malý problém byl \NL
ostrou pilou se ten problém snadno vyřešil \NL
Tak naposled se náš strom do nebe podíval \NL
a tupou ránu do větvoví už snad ani nevnímal 
\SLOKA Při stavbě se ukázalo že silnice bude dál \NL
a tak kousek od nový cesty smutnej pařez stál \NL
Dětem ani výletníkům z výšky nikdo nemával \NL
jen přítel vítr si o něm píseň na strništi z nouze hrál 
\REFREN  Jak tam stál… \NL
\end{TEXT}
