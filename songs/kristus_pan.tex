\begin{TEXT}{Kristus Pán}
\SLOKA	\Ch{C}{Do} kostela v \Ch{G}{džínách} občas \Ch{d}{chodí} Kristus \Ch{a}{Pán,}\NL
	\Ch{C}{do} copánku \Ch{G}{vlasy} občas \Ch{d}{nosí} Kristus \Ch{a}{Pán,} \NL
	\Ch{F}{v zimě} obut \Ch{G}{do} důchodek, \Ch{C}{v létě} v kristus\Ch{a}{kách,} \NL
	\Ch{d}{pře}kračuje \Ch{d7}{jist}ejch představ \Ch{G}{práh.}

\SLOKA	Denně scvrklou kůži oblíká si Kristus Pán,\NL
	po chodbách se vleče o dvou holích Kristus Pán.\NL
	Má v domově důchodců vše, co si může přát, \NL
	jak rád by měl navíc ještě rád.

\SLOKA	U běžících pásů taky sedá Kristus Pán,\NL
	do podzemních šachet taky fárá Kristus Pán,\NL
	koncem roku kvapem honí ohroženej plán, \NL
	upocenej dělník Kristus Pán.

\SLOKA	Na vysoký skály moc rád leze Kristus Pán,\NL
	pod širákem spává s partou trampů Kristus Pán,\NL
	záda rovný od usárny, šlape bláto cest,\NL
	dobře ví, kam každá musí vést.

\SLOKA	Poštu lidem nosí ve svý brašně Kristus Pán,\NL
	a ve vlacích lístky chodí štípat Kristus Pán, \NL
	a v moderně zařízenejch školních učebnách\NL
	od katedry hází na zeď hrách.

\SLOKA	Na saních rád jezdí jako vítr Kristus Pán,\NL
	na zamrzlý řece je vždy doma Kristus Pán,\NL
	dokáže si s vločkou sněhu na co chcete hrát,\NL
	dítětem se stává obzvlášť rád.
\end{TEXT}
