\begin{TEXT}{Sup a žluva}
\SLOKA Sed\Ch{E}{la} si stará \Ch{A}{žluva} na \Ch{E}{vy}ko\Ch{A}{tlaný} \Ch{E}{dub} \NL
Tu začal její \Ch{A}{půvab} ob\Ch{E}{divo}vati \Ch{H7}{sup} \NL
A \Ch{E}{takto} na ni \Ch{A}{mlu}ví: \uv{Oz\Ch{C#}{val} se ve mně \Ch{F#}{cit} \NL
a \Ch{H}{tvo}je oči \Ch{E}{žlu}ví chci \Ch{f#}{kaž}do\Ch{H7}{pádně} \Ch{E}{mít}.} 
\SLOKA \uv{Já nevím jak mám začít}, supovi jihne hlas \NL
\uv{Na tvojí hlavě ptačí nesmí být zkřiven vlas.} \NL
A žluva oči mhouří a sup hovoří dál: \NL
uv{za nejhroznějších bouří bych jako skála stál.} 
\SLOKA \uv{A kdyby snad vichr tvá hnízda na zem smet \NL		
A ty jsi byla mrtvá tak bych tě láskou sněd}\NL
Z toho poučení plyne pro žluvy \NL
Se supem o lásce že se nemluví 
\SLOKA /: \Ch{A (E,F#)}{Ať si žluvy} pozor dají :/ 3x \NL
\Ch{H}{Ať} si s námi zazpívají: \NL
\Ch{E}{Sup} sem sup tam \Ch{H7}{nám} už je to \Ch{E}{vše}chno jedno \NL
Sup sem sup tam \Ch{F#}{nám} už \Ch{H7}{je} to \Ch{E}{fuk\,!} \NL
\end{TEXT}
