\begin{TEXT}{Jonatán}
\SLOKA Když \Ch{G}{jaro} zaťukalo na dveře roku \Ch{C}{vosu}mnáctset\Ch{G}{šest} \NL
přišel \Ch{C}{vod} řeky chlap \Ch{G}{co} sem jel jen \Ch{a}{sám} a na svou \Ch{D}{pěst} \NL
na \Ch{G}{sobě} hrubej pytel vod kafe nosil \Ch{C}{místo} kabá\Ch{G}{tu} \NL
starý \Ch{C}{vosadníci} \Ch{G}{hádali} co k \Ch{a}{čertu} \Ch{D}{hledá} \Ch{G}{tu} 
\REFREN  Byl \Ch{e}{vazoun} děsný síly co sto šedesát mílí se \NL
/: \Ch{G(e)}{proti} proudu dřel :/ \NL
kam \Ch{A}{šel} ho všude chválej že jabloňovou álej \NL
/: pro \Ch{G(D)}{všechny} vysázel :/ \Ch{G}{ } \Ch{C}{ } \Ch{D}{ } \Ch{G}{ }
\SLOKA Dva měchy který složil v kantýně přivez' na dvou kánoích \NL
a jak v lokálu svý jméno řek' v tu ránu kdekdo ztich \NL
tam u nás bylo sice nezvyklý ale každej časem znal \NL
Jonatána co k nám jablečný jadýrka vozíval 
\SLOKA Když léty unavenej do trávy se svez a na zem sed \NL
vopřel se svejma zádama vo strom co právě kvet \NL
my spát pak nechali ho na místě kde právě přestal žít \NL
krásnější pomník nešel by snad vůbec postavit 
\SLOKA Až jednou nebudete na světě lidi ať tu po vás maj' \NL
/: třeba jabka nebo písničku co si rádi zazpívaj' :/ 
\end{TEXT}
