\begin{TEXT}{Hora Říp}
\REFREN  \Ch{G}{Už} je to dávno kroniky tvrdí \NL
\Ch{D}{tam} kde se tyčí hora Říp \NL
\Ch{A}{pár} lidí hledalo mléko a strdí \NL
\Ch{h}{zem} s vůní květů \Ch{G}{sta}letých \Ch{A}{lip} \NL
\Ch{G}{Kdy}koli z dálky vracíš se domů \NL
\Ch{D}{tak} jako tenkrát hledáš zas \NL
\Ch{A}{jak} dojít do země lipových stromů \NL
\Ch{h}{vždyť} jejich \Ch{G}{vů}ně \Ch{A}{zů}stala \Ch{D}{v nás} 
\SLOKA Dojít až \Ch{A}{tam} kde zrovna zraje \Ch{D}{réva} \NL
a člověk nena\Ch{A}{lévá} jen poloviční \Ch{D}{číš} \NL
dojít až \Ch{A}{tam} kde vyzvánějí \Ch{D}{k svá}tku \NL
a dětem pro po\Ch{A}{hádku} se v píseň promě\Ch{D}{níš} \NL
Na, \Ch{A}{na}, \Ch{D}{na}, \Ch{A}{na} \Ch{D}{…} 
\SLOKA Dojít až tam kde vlas babího léta \NL
všem stromům závoj splétá než uloží je spát \NL
dojít až tam kde písně kolovrátků \NL
ze starých bájí na památku \NL
všem lidem mohou hrát \NL
Na na na na… \NL
\end{TEXT}
