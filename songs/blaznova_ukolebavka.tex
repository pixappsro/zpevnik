\begin{TEXT}{Bláznova ukolébavka}
\SLOKA \Ch{D}{Máš,} má ovečko, \Ch{A}{dávno} spát, už pí\Ch{G}{seň} ptáků \Ch{D}{končí,}\NL
kvůli nám přestal i ví\Ch{A}{tr vát,} jen \Ch{G}{můra zírá} \Ch{D}{zvenčí,}\NL
já \Ch{A}{znám} její zášť, tak \Ch{G}{vyhledej} skrýš,\NL
zas \Ch{A}{má bílej} plášť a v \Ch{G}{okně} je \Ch{A}{mříž.}
\REFREN \Ch{D}{Máš,} má ovečko, \Ch{A}{dávno} spát,\NL
a mů\Ch{G}{žeš} hřát, ty mě \Ch{E}{můžeš} hřát,\NL
vždyť přij\Ch{D}{dou} se \Ch{G}{ptát,} zítra zas \Ch{D}{přijdou} se \Ch{G}{ptát,}\NL
jestli ty \Ch{D}{v mých} předsta\Ch{G}{vách} už \Ch{D}{mizíš.}
\SLOKA Máš, má ovečko, dávno spát, už máme půlnoc temnou,\NL
zítra budou nám bláznů lát, že ráda snídáš se mnou,\NL
proč měl bych jim lhát, že jsem tady sám,\NL
když tebe mám rád, když tebe tu mám.
\REFRENHRAJ
\end{TEXT}
